%Textformatierung
\documentclass[a4paper,bibliography=totocnumbered,12pt]{article}
\usepackage[top=1.5cm, left=2cm, right=2cm, bottom=1.5cm, includeheadfoot]{geometry}
\usepackage[utf8]{inputenc}
\usepackage{setspace}
\usepackage[nouppercase]{scrlayer-scrpage}

\usepackage{ngerman}

%Appendix
\usepackage[toc,page]{appendix}

%Mathematik
\usepackage{amsmath}
\allowdisplaybreaks
\usepackage{amsfonts}
\usepackage{amssymb}
\usepackage{verbatim}
\usepackage{slashed}
\usepackage{braket}
\usepackage{dsfont}
\usepackage{subfigure}
\usepackage{array,booktabs}
\usepackage[framemethod=tikz]{mdframed}
\mdtheorem[
  linecolor=purple,
  frametitlefont=\sffamily\bfseries\color{white},
  frametitlebackgroundcolor=purple,
]{Def}{Formül}


%Bildverwaltung
\usepackage{graphicx}
\usepackage{grffile}
\usepackage[export]{adjustbox}
%\usepackage{subcaption}
%Größe der 3-Nebeneinander-Graphen
\newcommand{\gSize}{6.3cm}
\usepackage[strict]{changepage}
\usepackage{multirow}
%Horizontale Linie
\newcommand{\HRule}{\rule{\linewidth}{0.5mm}}
%\usepackage{auto-pst-pdf} 
%Quellcode
\usepackage{listings}
\usepackage{color}
\definecolor{light-gray}{gray}{0.5}
\definecolor{greenly}{rgb}{0,0.2,0}
\usepackage[width=0.8\textwidth,labelfont=bf,font=small]{caption}
\usepackage{xcolor,colortbl}

\lstdefinestyle{customc}{
  belowcaptionskip=1\baselineskip,
  breaklines=true,
  frame=tb,
  numbers=left,
  numberstyle=\color{light-gray},
  xleftmargin=\parindent,
  language=C,
  showstringspaces=false,
  %code-Farben
  basicstyle=\footnotesize\ttfamily,
  commentstyle= \color{light-gray},
  keywordstyle= \color{blue},
  stringstyle= \color{greenly}
}

\lstset{style = customc}

\usepackage{sectsty}
%\chapternumberfont{\Large} 
%\chaptertitlefont{\Large}
\usepackage{titlesec}

%\titleformat{\section}
%  {\normalfont\large\bfseries}{\thesection}{1em}{}
  
%\titleformat{\subsection}
%  {\normalfont\large\bfseries}{\thesubsection}{1em}{}


%\makeatletter
%\def\@makechapterhead#1{%
%  \vspace*{50\p@}%
%  {\parindent \z@ \raggedright \normalfont
 %   \ifnum \c@secnumdepth >\m@ne
  %    \if@mainmatter
        %\huge\bfseries \@chapapp\space \thechapter
   %     \Large\bfseries \thechapter \space \space%
        %\par\nobreak
        %\vskip 20\p@
    %  \fi
 %   \fi
  %  \interlinepenalty\@M
  %  \Large \bfseries #1\par\nobreak
  %  \vskip 20\p@
 % }}
%\makeatother

\setkomafont{pagehead}{%
\normalfont}

%\renewcommand*\chaptermarkformat{\thechapter. } 

\lstset{breaklines=true,
  breakatwhitespace=true,
  stepnumber=1,
  basicstyle=\ttfamily\footnotesize,
  commentstyle=\ttfamily\color{gray},
  prebreak={\textbackslash},
  breakindent=10pt,
  breakautoindent=false,
  showspaces=false,
  showstringspaces=false,
  frame=shadowbox,
  rulesepcolor=\color{gray},
  rulesep=0.1em,
  abovecaptionskip=0em,
  aboveskip=1.5em,
  belowcaptionskip=0.5em,
  belowskip=1em,
}
\makeatletter
\lstnewenvironment{numlstlisting}[1][]{%
  \lstset{%
    #1,
    numbers=left,
    firstnumber=auto,
    numberstyle=\tiny\sffamily}%
  \csname\@lst @SetFirstNumber\endcsname
}{%
  \csname \@lst @SaveFirstNumber\endcsname
}
\usepackage{german}

% Zitieren 
%AB HIER FÜGE ICH HINZU UND MÖCHTE Zitierweise wie ÖKonomen (Author Jahr), aber der macht wie 
%Naturwissenschaftler [1]...wenn du ab hier als kommentar machst, dann
\usepackage[style=authoryear, backend=bibtex, bibencoding=utf-8, language=\ifnum\value{Language}=1 german\else english\fi, block=space, uniquename=init, giveninits=true, maxcitenames=2, maxbibnames=99, dashed=false, isbn=false, doi=false, date=year]{biblatex}
\bibliography{.bib}												% Name of the database file
	% Abstand
	\setlength{\bibitemsep}{6pt}
	% Autor und Jahr werden im Literaturverzeichnis fett geschrieben
	\xpretobibmacro{author}{\mkbibbold\bgroup}{}{}
	\xapptobibmacro{author}{\egroup}{}{}
	\xpretobibmacro{bbx:editor}{\mkbibbold\bgroup}{}{}
	\xapptobibmacro{bbx:editor}{\egroup}{}{}
	% Doppelpunkt statt Punkt hinter author-year
	\renewcommand*{\labelnamepunct}{\mkbibbold{\addcolon\space}}
	% Autoren durch Semikolon getrennt
	\AtBeginBibliography{%
	\renewcommand*{\multinamedelim}{\addsemicolon\space}%
	\renewcommand*{\finalnamedelim}{\addsemicolon\space}%
	}
	% Keine Anführungszeichen in der Bibliographie
	\DeclareFieldFormat[article,inbook,incollection,inproceedings,patent,thesis,unpublished]{title}{#1\isdot}
	% Journal: Nummer in Klammern
	\renewbibmacro*{journal+issuetitle}{
		\usebibmacro{journal}\setunit*{\addspace}\iffieldundef{series}{}
		{\newunit \printfield{series}\setunit{\addspace}}\printfield{volume}\iffieldundef{number}{}
		{\mkbibparens{\printfield{number}}}\setunit{\addcomma\space}\printfield{eid}\setunit{\addspace}\usebibmacro{issue+date}\setunit{\addcolon\space}\usebibmacro{issue}\newunit
	}
	% In ngerman: et al. statt u.a.
	\DefineBibliographyStrings{ngerman}{andothers = {{et\,al\adddot}}}

% Formatierung Literaturverzeichnis
\apptocmd{\UrlBreaks}{\do\f\do\m}{}{}
\setcounter{biburllcpenalty}{9000}% Kleinbuchstaben
\setcounter{biburlucpenalty}{9000}% Großbuchstaben

% Trennhinweise
\hyphenation{Ma-na-ge-ment Netz-werk}

%BIS HIER HIN IST DAS NEU
%%%%%%%%%%%%%%%%%%%%%%%%%%%%%%%%%%%%%%%%%%%%%
\begin{document}
\pagenumbering{roman}
\begin{titlepage}
	\centering
	{\textcolor{white}{nichts}\par}
	\vspace{2cm}
	{\bfseries Exposé \par}
	\vspace{0.1cm}
	{über das geplante Promotionsvorhaben \par}
	\vspace{0.1cm}
	{mit dem Titel \par}
	\vspace{3cm}
{\bfseries\Large (Beispiel) Three Essays on \\ Market and Mechanism Design \par}
	\vspace{3cm}
%{in der Arbeitsgruppe Prof.\ Dr.\ Ingrid\ Ott\par}
	%\vspace{0.1cm}$
	{unter der wissenschaftlichen Betreuung von Prof.\ Dr.\ Ingrid\ Ott\par}
	\vspace{0.1cm}
	{am Institut für Volkswirtschaftslehre (ECON)\par}
	\vspace{0.1cm}
	{der Fakultät für Wirtschaftswissenschaften \par}
	\vspace{0.1cm}
	{des Karlsruher Instituts für Technologie\par}
	\vspace{3cm}
	{\bfseries\Large Resul Zoro\u{g}lu \par}
	\vspace{0.1cm}

	\vfill

% Bottom of the page
	{13. September 2021}
\end{titlepage}
%Seitennummer nicht Anzeigen
\thispagestyle{empty}


\clearpage
\thispagestyle{empty}
\clearpage

\pagestyle{scrheadings}
\clearscrheadfoot
\ohead{\headmark}
%\automark[section]{chapter}
\ifoot{}
\cfoot{}
\ofoot{\pagemark}
\tableofcontents
\clearpage
\thispagestyle{empty}
\clearpage
\pagenumbering{arabic}

\section{Hintergrund}
Volkswirtschaftslehre ist die Sozialwissenschaft, die untersucht, wie Menschen mit ökonomischen Werten interagieren, insbesondere der Produktion, der Verteilung und dem Verbrauch von Waren und Dienstleistungen \cite{krugmanEconomics2013}. \\
Der ökonomische Wert ist ein Maß für den Nutzen, den eine Ware oder eine Dienstleistung einem Menschen bringt. Meist wird der Wert an der Zahlungsbereitschaft gemessen und ist folgendermaßen zu interpretieren: „Wie viel ist der Mensch bereit für die Ware oder Dienstleistung zu zahlen?“.

Letztere Fragestellung lässt sich in Märkten ohne Preissystem schwer beantworten. Dieser Problemstellung gehen Ökonomen auf den Grund, insbesondere im Bereich des Markt und des Mechanism Designs. Beispielsweise haben Lloyd Shapley und Alvin Roth soziale Allokationsprobleme, bei denen Geld als Zahlungsmittel ausgeschlossen bzw. eingeschränkt ist, untersucht und den Nobelpreis für ihre Werke „Theorien stabiler Allokationen und die Praxis der Marktgestaltung“ erhalten. Reale Matching-Probleme sind beispielsweise Organtausch \cite{rothKidneyExchange2004a} und die Zuweisung von Einwohnern an ländliche Krankenhäuser \cite{rothAllocationResidentsRural1986}.

Die Volkswirtschaftslehre ist in Mikro- und Makroökonomie zu gliedern und die Zuordnung des Markt und Mechanism Design erfolgt in die Mikroökonomie.  \\ 
Allgemein formuliert untersucht die Mikroökonomie das Verhalten von Einzelpersonen und Unternehmen bei Entscheidungen über die Verteilung knapper Ressourcen und Wechselwirkungen ziwschen den Wirtschaftssubjekten \cite{marchant1991macroeconomic}. \\
Die Makroökonomie hingegen konzentriert sich auf die Gesamtheit der Wirtschaftstätigkeit und befasst sich mit Fragen des Wachstums, der Inflation und der Arbeitslosigkeit sowie mit der nationalen Politik in Bezug auf diese Themen \cite{mankiw2003macroeconomics}. \\


Der Ökonom Patrick Legros und die Ökonomin Estelle Cantillion betonen sogar, dass Mechanism Design zum Standardwerkzeug eines jeden Wirtschaftswissenschaftlers gehört, und jeder Wirtschaftswissenschaftler verwendet sie - bewusst oder unbewusst - fast täglich. Sie abstrahieren Mechanism Design auf zwei Ebenen \cite{legrosNobelPrizeWhat2007}:
	\begin{itemize}
	\item	Mechanism Design gibt Auskunft, wann Märkte oder marktbasierte Institutionen wahrscheinlich zu den wünschenswerten Ergebnissen führen und wann sich andere Institutionen besser eignen
	\item	Mechanism Design gibt eine Orientierung, wenn Märkte versagen, wie alternative Institutionen zu gestalten sind.
	\end{itemize}


\newpage
\section{Die Dissertation}
\subsection{Ziel der Dissertation}

Lorem ipsum dolor sit amet, consectetuer adipiscing elit, sed diam nonummy nibh euismod tincidunt ut laoreet dolore magna aliquam erat volutpat. Ut wisi enim ad minim veniam, quis nostrud exerci tation ullamcorper suscipit lobortis nisl ut aliquip ex ea commodo consequat. Duis autem vel eum iriure dolor in hendrerit in vulputate velit esse molestie consequat, vel illum dolore eu feugiat nulla facilisis at vero et accumsan et iusto odio dignissim qui blandit praesent luptatum zzril delenit augue duis dolore te feugait nulla facilisi. Lorem ipsum dolor sit amet, consectetuer adipiscing elit, sed diam nonummy nibh euismod tincidunt ut laoreet dolore magna aliquam erat volutpat. Ut wisi enim ad minim veniam, quis nostrud exerci tation ullamcorper suscipit lobortis nisl ut aliquip ex ea commodo consequat. Duis autem vel eum iriure dolor in hendrerit in vulputate velit esse molestie consequat, vel illum dolore eu feugiat nulla facilisis at vero et accumsan et iusto odio dignissim qui blandit praesent luptatum zzril delenit augue duis dolore te feugait nulla facilisi. Nam liber tempor cum soluta nobis eleifend option congue nihil imperdiet doming id quod mazim placerat facer possim assum.

\subsection{Theorien und Methoden}
Lorem ipsum dolor sit amet, consectetuer adipiscing elit, sed diam nonummy nibh euismod tincidunt ut laoreet dolore magna aliquam erat volutpat. Ut wisi enim ad minim veniam, quis nostrud exerci tation ullamcorper suscipit lobortis nisl ut aliquip ex ea commodo consequat. Duis autem vel eum iriure dolor in hendrerit in vulputate velit esse molestie consequat, vel illum dolore eu feugiat nulla facilisis at vero et accumsan et iusto odio dignissim qui blandit praesent luptatum zzril delenit augue duis dolore te feugait nulla facilisi. Lorem ipsum dolor sit amet, consectetuer adipiscing elit, sed diam nonummy nibh euismod tincidunt ut laoreet dolore magna aliquam erat volutpat. Ut wisi enim ad minim veniam, quis nostrud exerci tation ullamcorper suscipit lobortis nisl ut aliquip ex ea commodo consequat. Duis autem vel eum iriure dolor in hendrerit in vulputate velit esse molestie consequat, vel illum dolore eu feugiat nulla facilisis at vero et accumsan et iusto odio dignissim qui blandit praesent luptatum zzril delenit augue duis dolore te feugait nulla facilisi. Nam liber tempor cum soluta nobis eleifend option congue nihil imperdiet doming id quod mazim placerat facer possim assum.

\subsection{Aktueller Forschungsstand}
Lorem ipsum dolor sit amet, consectetuer adipiscing elit, sed diam nonummy nibh euismod tincidunt ut laoreet dolore magna aliquam erat volutpat. Ut wisi enim ad minim veniam, quis nostrud exerci tation ullamcorper suscipit lobortis nisl ut aliquip ex ea commodo consequat. Duis autem vel eum iriure dolor in hendrerit in vulputate velit esse molestie consequat, vel illum dolore eu feugiat nulla facilisis at vero et accumsan et iusto odio dignissim qui blandit praesent luptatum zzril delenit augue duis dolore te feugait nulla facilisi. Lorem ipsum dolor sit amet, consectetuer adipiscing elit, sed diam nonummy nibh euismod tincidunt ut laoreet dolore magna aliquam erat volutpat. Ut wisi enim ad minim veniam, quis nostrud exerci tation ullamcorper suscipit lobortis nisl ut aliquip ex ea commodo consequat. Duis autem vel eum iriure dolor in hendrerit in vulputate velit esse molestie consequat, vel illum dolore eu feugiat nulla facilisis at vero et accumsan et iusto odio dignissim qui blandit praesent luptatum zzril delenit augue duis dolore te feugait nulla facilisi. Nam liber tempor cum soluta nobis eleifend option congue nihil imperdiet doming id quod mazim placerat facer possim assum.

\section{Inhaltliche Gliederung der Dissertation}
\begin{enumerate}
	\item Einleitung
	\item Erläutern relevanter Theorien
	%\begin{itemize}
		%\item Standardmodell der Teilchenphysik
		%\item Supersymmetrie
		%\item Composite Higgs Models
		%\item Flavourverletzung
		%\begin{itemize}
		%	\item minimal flavour violation und non-minimal flavour violation
		%	\item dark minimal flavour violation framework
		%\end{itemize}
	%\end{itemize}
	\item Thema 1
	%\begin{itemize}
		%\item fermionische Top-Partner 
		%\item bosonische Top-Partner 
		%\item dunkle Materie mit Top-Flavour
	%\end{itemize}
	\item Thema 2
	%\begin{itemize}
		%\item Reinterpretation existierender experimenteller Schranken
		%\item Analyse neuer flavourverletzender Signaturen
		%\item Gegen\"uberstellung der Ergebnisse aus den verschiedenen Modellen
	%\end{itemize}
	\item Thema 3
	\item Zusammenfassung und Ausblick
\end{enumerate}
\clearpage


\section{Zeitplan}

\subsubsection*{Einarbeitungsphase (ca. 4 Monate)}

In der Einarbeitungsphase werden die Kenntnisse in der theoretischen Wirtschaftspolitik im Hinblick auf das Dissertationsthema vertieft. Insbesondere wird die vorhandene Literatur des Themenkomplexes gesichtet und studiert.  

\subsubsection*{Analysephase (ca. 24 Monate, davon je Modell ca. 8 Monate)}

Die Analysephase stellt den Hauptteil des Promotionsvorhabens dar. 

\subsubsection*{Publikationsphase (ca. 4 Monate)}

In der Publikationsphase werden die erzielten Ergebnisse in wissenschaftlichen Fachzeitschriften publiziert. Zudem sollen die Resultate auf internationalen Konferenzen vorgestellt werden.

\subsubsection*{Abschlussphase (ca. 4 Monate)}

In der Abschlussphase wird die Dissertationsschrift verfasst und vor der Fakult\"at f\"ur Wirtschaftswissenschaften Instituts f\"ur Technologie verteidigt.

\clearpage

%\phantomsection
%\addcontentsline{toc}{chapter}{Literaturverzeichnis}
%\nocite{*}

\renewcommand{\refname}{Literaturverzeichnis}
\bibliography{bib}
\addcontentsline{toc}{section}{Literaturverzeichnis}
\bibliographystyle{JHEP}

%\printbibliography[
%	title={\ifnum\value{Language}=1 Literaturverzeichnis \else References \fi},
%	heading=bibintoc
%	]

%\backmatter 

%\printbibliography[heading=head]
\end{document}
